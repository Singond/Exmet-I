% Copyright 2021, 2022 Radek Hornak, Jan Slany, Lukas Vrana
\documentclass{protokol}

\usepackage[czech]{babel}
\usepackage[utf8]{inputenc}
\usepackage{icomma}

% Plovouci bloky (obrazky, tabulky)
\usepackage{floatrow}
\floatsetup[table]{capposition=top}
\floatsetup[figure]{frameset={\fboxsep16pt}}
\usepackage{subcaption}

% Tabulky
\usepackage{tabu}
\usepackage{booktabs}
\usepackage{csvsimple}
\usepackage{multirow}
\usepackage{multicol}

% Jednotky
\usepackage{siunitx}
\sisetup{
	locale               = DE,
	inter-unit-product   = \ensuremath{{}\cdot{}},
	list-units           = single,
	list-separator       = {; },
	list-final-separator = \text{ a },
	list-pair-separator  = \text{ a },
	range-phrase         = \text{ až },
	range-units          = single,
	separate-uncertainty = true,
}
\usepackage{amsmath}

% Obvody
\usepackage{circuitikz}

% Obrazky a grafy
% \usepackage{graphicx}
\graphicspath{
	{img/}
	{plots/}
	{build/plots/}
}
\usepackage{epstopdf}
\epstopdfsetup{outdir=./build/plots/}

\usepackage[backend=biber, sorting=none, sortlocale=cs_CZ]{biblatex}
\addbibresource{references.bib}

\jmenopraktika={Experimentální metody I}       % jmeno predmetu
\jmeno={Radek Horňák, Jan Slaný, Lukáš Vrána}  % jmeno mericiho
\obor={F}                                      % zkratka studovaneho oboru
\skupina={Čt 15:00}                            % doba vyuky seminarni skupiny
\rocnik={IV}
\semestr={I}

\cisloulohy={01}
\jmenoulohy={Vektorová síťová analýza}

\datum={22. září 2021}                  % datum mereni ulohy
\tlak={}% [hPa]
\teplota={}% [C]
\vlhkost={}% [%]

\newcommand\sparam{S}
\newcommand\male{m}
\newcommand\female{f}
\newcommand\permitfree{\varepsilon_0}
\newcommand\permitrel{\varepsilon_r}
\newcommand\permeabfree{\mu_0}
\newcommand\freq{f}
\newcommand\impedance{Z}
\newcommand\resistance{R}
\newcommand\inductance{L}
\newcommand\capacitance{C}
\newcommand\wavelen{\lambda}
\newcommand\real{\operatorname{Re}}
\newcommand\imag{\operatorname{Im}}

\newcommand\connector[2]{#1 -- #2}
\newcommand\connectord[3]{#1 -- #2\\ #3}
\tikzset{
	connector/.style={
		draw,
		align=center,
		minimum width=4cm,
		minimum height=1.5cm}
}

\begin{document}
\headernoenv

\section{Úvod}
Elektrické obvody pracující na nízkých frekvencích,
jejichž rozměry jsou malé oproti vlnové délce signálu,
mají v~každém bodě jasně definované napětí a proud.
Díky malým rozměrům je v~nich zároveň zanedbatelné fázové zpoždění
mezi jednotlivými body v~obvodu.
Tyto vlastnosti však obecně neplatí pro vysokofrekvenční obvody.
K~jejich analýze je potřeba přistupovat odlišným způsobem,
než jaký se využívá pro nízkofrekvenční obvody.
V~této úloze se budeme zabývat vektorovou síťovou analýzou vysokofrekvenčních obvodů.

\section{Teorie}

\subsection{Vektorový síťový analyzátor}

Následující teoretická část včetně terminologie a značení vychází zejména
z~knihy \emph{Microwave Engineering} \cite{pozar}.
Vektorový síťový analyzátor (VNA, z~anglického \emph{vector network analyzer})
je zpravidla dvoukanálový mikrovlnný měřicí přístroj,
sloužící k~měření rozptylových parametrů vysokofrekvenčních elektrických obvodů.
Umožňuje měřit velikost a fázi vlnění, které obvodem prochází
či se od něj odráží.
Zkoumaný obvod bývá někdy označován zkratkou DUT
(z~anglického \emph{device under test}).

VNA obsahuje zdroj signálu o~laditelné frekvenci a soustavu přijímačů,
které měří odražený a~procházející signál.
Schéma funkce přístroje je na obr.~\ref{VNA}.
Vygenerovaný signál postupuje směrem k~DUT.
Část signálu se na vstupu DUT odrazí zpět, část projde k~výstupní bráně čili portu.
Přijímače porovnávají výsledný signál na ně dopadající
se signálem vygenerovaným ve zdroji.
Výsledek je poté počítačově zpracován a zobrazen na displeji.
Výstupem měření jsou prvky rozptylové matice.

\begin{figure}[b]
	\centering
	\includegraphics[width=100mm,]{network-analyzer-diagram}
	\caption{Princip funkce VNA, převzato z~\cite{tektronix} a upraveno.}
	\label{VNA}
\end{figure}


\subsection{Rozptylová matice}

V mikrovlnných obvodech dokážeme dobře měřit výkon, impedanci a fázi. 
Napětí a proudy nejsou v~tomto případě na rozdíl od nízkofrekvenčních obvodů
dobře definovány.
Zavádíme pojem \emph{rozptylová matice}, také označovaná jako $S$ matice.
Rozptylová matice poskytuje kompletní popis obvodu s~$N$ branami,
konkrétně popisuje vztah mezi vlněním dopadajícím na brány
a vlněním na branách odraženým.
Pro některé prvky je možné $S$ matici spočítat,
ovšem často jednodušší způsob je prvky matice přímo změřit pomocí VNA.
% TODO: Nedefinovat pomoci napeti, kdyz to neni dobre definovano
$S$ matice je definována vztahem:
\begin{equation}
	\label{eq:smatrix-short}
	V^- = S V^+
\end{equation}
kde $V^+$ a $V^-$ jsou amplitudy dopadající a odražené vlny
(v~tomto pořadí).

Pro obvod s~$N$ branami má $S$ matice tvar:
\begin{equation}
	\label{eq:smatrix-full}
	\begin{pmatrix}
		V_1^-  \\
		V_2^-  \\
		\vdots \\
		V_N^-
	\end{pmatrix}
	=
	\begin{pmatrix}
		S_{11} & S_{12} & \dots & S_{1N} \\
		S_{21} &        &       & \vdots \\
		\vdots &        &       & \vdots \\
		S_{N1} & \dots	& \dots & S_{NN} \\
	\end{pmatrix}
	%
	\begin{pmatrix}
		V_1^+  \\
		V_2^+  \\
		\vdots \\
		V_N^+
	\end{pmatrix}
\end{equation}

Pro obvod s~dvěma branami tedy bude $S$ matice řádu 2.
V~ní prvek $S_{11}$ je spojen s odraženým signálem na vstupu,
$S_{22}$ s odrazem na výstupu, $S_{21}$ je transmisní prvek ze vstupu na výstup
a $S_{12}$ je transmisní prvek z~výstupu na vstup.

\subsection{Impedanční a admitanční matice}

Impedance $Z$ je komplexní veličinou a je definovaná jako poměr napětí a proudu:
\begin{equation}
	\label{eq:impedance}
	Z~= \frac{U}{I}
\end{equation}
Impedanční matice mikrovlnného obvodu s~$N$ branami
udává vztah mezi napětími a proudy následovně:
\begin{equation}
	\label{eq:zmatrix}
	\begin{pmatrix}
		V_1    \\
		V_2    \\
		\vdots \\
		V_N
	\end{pmatrix}
	=
	\begin{pmatrix}
		Z_{11} & Z_{12} & \dots & Z_{1N} \\
		Z_{21} &        &       & \vdots \\
		\vdots &        &       & \vdots \\
		Z_{N1} & \dots	& \dots & Z_{NN} \\
	\end{pmatrix}
	%
	\begin{pmatrix}
		I_1    \\
		I_2    \\
		\vdots \\
		I_N
	\end{pmatrix}
\end{equation}
Admitance $Y$ je převrácenou hodnotou impedance, tedy platí:
\begin{equation}
	\label{eq:admittance}
	Y = \frac{1}{Z}
\end{equation}
Admitanční matice je definovaná jako:
\begin{equation}
	\label{eq:ymatrix-short}
	Y = Z^{-1}
\end{equation}
což pro obvod s~$N$ branami můžeme rozepsat ve tvaru:
\begin{equation}
	\label{eq:ymatrix-full}
	\begin{pmatrix}
		I_1    \\
		I_2    \\
		\vdots \\
		I_N
	\end{pmatrix}
	=
	\begin{pmatrix}
		Y_{11} & Y_{12} & \dots & Y_{1N} \\
		Y_{21} &        &       & \vdots \\
		\vdots &        &       & \vdots \\
		Y_{N1} & \dots	& \dots & Y_{NN} \\
	\end{pmatrix}
	%
	\begin{pmatrix}
		V_1    \\
		V_2    \\
		\vdots \\
		V_N
	\end{pmatrix}
\end{equation}

\subsection{Výpočet impedanční a admitanční matice z~rozptylové matice}

Impedanční matici můžeme vyjádřit pomocí rozptylové matice jako:
\begin{equation}
	\label{eq:zmatrix-smatrix}
	Z = [E + S] [E - S]^{-1}
\end{equation}
kde $E$ je jednotková matice.
Obdobně můžeme pomocí rozptylové matice a jednotkové matice
vyjádřit i admitanční matici:
\begin{equation}
	\label{eq:ymatrix-smatrix}
	Y = [E - S] [E + S]^{-1}
\end{equation}

\subsection{Transmisní matice}
Při měření obvod zpravidla tvoří série prvků.
Při výpočtech sériově zapojených prvků je vhodné zavést
\emph{transmisní matici} $T$ pro každý z~nich.
U~dvojbranů je matice $T$ řádu 2 a má tvar:
\begin{equation}
	\label{eq:tmatrix}
	\begin{pmatrix}
		A & B  \\
		C & D  \\
	\end{pmatrix}
\end{equation}

V~nejjednodušším obvodu se dvěma prvky je mezi celkovou transmisní
maticí $ABCD$, transmisní maticí prvního prvku $A_1B_1C_1D_1$
a transmisní maticí druhého prvku $A_2B_2C_2D_2$ vztah následující:
\begin{equation}
	\label{eq:tmatrix-series}
	\begin{pmatrix}
		A & B  \\
		C &	D  \\
	\end{pmatrix}
	=
	\begin{pmatrix}
		A_1 & B_1  \\
		C_1 & D_1  \\
	\end{pmatrix}
	%
	\begin{pmatrix}
		A_2 & B_2  \\
		C_2 & D_2  \\
	\end{pmatrix}
\end{equation}

Dále platí:
\begin{align}
	\label{eq:tmatrix-elema}
	A &= \frac{(1+S_{11})(1-S_{22})+S_{12}S_{21}}{2S_{21}} \\
	\label{eq:tmatrix-elemb}
	B &= \frac{(1+S_{11})(1+S_{22})-S_{12}S_{21}}{2S_{21}} Z_0 \\
	\label{eq:tmatrix-elemc}
	C &= \frac{(1-S_{11})(1-S_{22})-S_{12}S_{21}}{2S_{21}} \frac{1}{Z_0} \\
	\label{eq:tmatrix-elemd}
	D &= \frac{(1-S_{11})(1+S_{22})+S_{12}S_{21}}{2S_{21}}
\end{align}
kde $\impedance_{0}$ je charakteristická impedance.
Zpětný přepočet prvků $A$, $B$, $C$, $D$ na $S_{11}$ a $S_{21}$
lze provést následovně:
\begin{align}
	\label{eq:s11}
	S_{11} &= \frac{A+B/Z_0-CZ_0-D}{A+B/Z_0+CZ_0+D} \\
	\label{eq:s21}
	S_{21} &= \frac{2}{A+B/Z_0+CZ_0+D}
\end{align}

\subsection{Výpočet impedance z~$\sparam_{11}$ v~případě jednobranu}
V~případě jednobranu měříme pouze prvek $\sparam_{11}$,
z~nějž lze impedanci $\impedance$ dopočítat podle vztahu:
\begin{equation}
	\impedance = \impedance_{0}\frac{1+\sparam_{11}}{1-\sparam_{11}}
	\label{eq:Z}
\end{equation}


\subsection{Typy konektorů}
V~tomto praktiku budeme měřit různá zapojení koaxiálních konektorů typu
SMA, BNC a N s~impedancí 50~$\Omega$.

Konektor \emph{SubMiniature Type-A} neboli SMA je jedním z~nejpoužívanějších
mikrovlnných konektorů. Je zobrazený na obr.~\ref{SMA}.
Průměry jeho koaxiálního vedení jsou 4,1~mm a 1,27~mm,
funkci izolantu v~něm plní teflon.
Spojení se zajišťuje pomocí šroubovacího mechanismu.
Obecně se uvádí, že je vhodný až do frekvence 18~GHz \cite{rfhandbook}.

\emph{Bayonet Neill--Concelman}, někdy nazývaný \emph{Bayonet Naval Connector},
zkráceně BNC, je konektor vyvinutý ve čtyřicátých letech dvacátého století.
Jeho charakteristikou je bajonetový mechanismus spojení.
Tento mechanismus je tvořen dvěma výstupky na samičím konektoru
a drážkou se spirálou a vybráním na převlečné části samčího konektoru,
viz obr.~\ref{BNC}.
Umožňuje rychlé a spolehlivé spojení nasunutím a pootočením převlečné
části o~90$^{\circ}$ \cite{czwiki}.
Dle \cite{rfhandbook} je konektor vhodný až do frekvence 11~GHz,
většinou je však uváděna hodnota nižší.

Konektor typu N, pojmenovaný po jeho vynálezci Paulu Neillovi,
vznikl ve čtyřicátých letech dvacátého století.
Jedná se o~konektor s~klasickým šroubovacím mechanismem.
Využívá se při aplikacích, kde frekvence nepřesahuje 11~GHz \cite{rfhandbook}.
Je zobrazený na obr.~\ref{N}.

\begin{center}
	\captionsetup{justification=centering}
	\begin{minipage}{0.32\textwidth}
		\includegraphics[width=\textwidth]{connector-sma}
		\captionof{figure}{SMA konektor, převzato z~\cite{gme}}
		\label{SMA}
	\end{minipage}
	\begin{minipage}{0.40\linewidth}
		\includegraphics[width=\linewidth]{connector-bnc}
		\captionof{figure}{BNC konektor, převzato z~\cite{czwiki}}
		\label{BNC}
	\end{minipage}
	\begin{minipage}{0.29\textwidth}
		\includegraphics[width=\textwidth]{connector-n}
		\captionof{figure}{N konektor, převzato z~\cite{enwiki}}
		\label{N}
	\end{minipage}
\end{center}

\section{Praktická část}
\subsection{Kalibrace přístroje}
K~měření budeme používat vektorový síťový analyzátor Rohde \& Schwarz ZVL,
který měří v~rozsahu 9~kHz až 13,6~GHz.
Měřený prvek se k~přístroji připojuje pomocí testovacích kabelů s~SMA konektory.
Tyto kabely však nejsou dokonalé a mění parametry celého systému.
Před samotným měřením je tedy potřeba přístroj s~kabely nakalibrovat.
Pro kalibraci jedné brány potřebujeme tři různé impedance.
Z~praktických důvodů se volí známé impedance při třech situacích -- zkrat (short),
otevřený obvod (open) a přizpůsobený obvod (match).
Nejvhodnějším způsobem kalibrace je využití kalibračního členu,
pomocí nějž lze jednoduše výše zmíněné impedance vytvořit.
K~dispozici máme kalibrační člen Rohde \& Schwarz ZV-Z135,
k~němuž jsou dodávány i korekce, které už máme ve VNA nahrané.
Kalibraci je potřeba provést na každé bráně zvlášť i na obou branách současně.
Zároveň je potřeba kalibraci provést pečlivě,
protože nedbalá kalibrace způsobená špatně připojeným kalibračním členem
by mohla ovlivnit všechna následně naměřená data.

První připojíme bránu 1 jako otevřený obvod.
Veškerý signál by se měl odrazit, čemuž by mělo odpovídat $|S_{11}|$ = 0~dB.
Další na řadě je zkrat na bráně 1.
Na zkratu se nemá výkon kde ztratit,
takže by opět mělo docházet k~úplnému odrazu a platit, že $|S_{11}|$ = 0~dB.
Zbývá připojení brány 1 na přizpůsobenou impedanci.
V~tomto případě by se měl ideálně veškerý signál absorbovat,
tedy $|S_{11}|$ by měla být co nejnižší.
U našeho kalibračního členu je na frekvenci 10~GHz $|S_{11}|$ přibližně $-25$~dB,
minimum $|S_{11}|$ je okolo $-50$~dB. Tyto hodnoty považujeme
za dostatečně nízké.

Kalibrace druhé brány probíhá obdobně,
přičemž se místo $S_{11}$ měří $S_{22}$.
Pro otevřený obvod, zkrat i přizpůsobený obvod dává přístroj prakticky
stejné hodnoty $S_{22}$ na bráně 2 jako jsme naměřili pro $S_{11}$ na bráně 1.

Posledním krokem je kalibrace obou bran současně nazývaná
jako zapojení through.
Přístroj měří $S_{21}$, jehož velikost je podle očekávání kolem 0~dB.
Tím je kalibrace hotová.

Z~předpokladu symetrie budeme dále uvažovat rovnosti
$S_{11}$ = $S_{22}$ a $S_{21}$ = $S_{12}$.

\subsection{Měření 1: zlatá spojka SMA{\female} -- SMA{\female}}
Jako první měříme spojku SMA, která má z~obou stran samičí port (značíme f),
schéma zapojení je na obr.~\ref{fig:exp1}.
Spojka je zlaté barvy (index $_\text{Z}$), což rozlišujeme pouze z~důvodu,
že budeme srovnávat dvě spojky stejného typu lišící se barvou.
Výsledné zapojení tedy označíme jako (SMAf/SMAf)$_\text{Z}$.
Obdobné značení budeme využívat i nadále.
Z~obr.~\ref{fig:01-sparam} je vidět,
že v~GHz frekvencích se na spojce začíná signál odrážet (viz $\sparam_{11}$),
a na~frekvencích blížících se \SI{10}{\giga\hertz} je signál zeslaben téměř
o~\SI{0.5}{\decibel}.

\begin{figure}[h]
	\centering
	\begin{circuitikz}
		\node[connector] (con1) at (0,0)
		{\connectord{SMA\female}{SMA\female}{zlatý}};
		\draw (con1.west) to[short, -o] +(-1,0);
		\draw (con1.east) to[short, -o] +(1,0);
	\end{circuitikz}
	\caption{Schéma zapojení 1. měření.}
	\label{fig:exp1}
\end{figure}

\subsection{Měření 2: stříbrná spojka SMA{\female} -- SMA{\female}}
Měříme opět spojku (SMAf/SMAf)$_\text{S}$, avšak nyní je stříbrné barvy (index $_\text{S}$).
Schéma zapojení je na obr.~\ref{fig:exp2}.
V~porovnání se spojkou ve zlaté barvě z~prvního měření má $S_{11}$ stříbrné spojky na obr.~\ref{fig:02-sparam} podobný průběh.
U~$S_{21}$ je vidět oproti zlaté spojce menší útlum.
Může se tedy jednat o~kvalitnější součástku,
nebo je lepší průchod signálu způsoben preciznějším dotažením.

\begin{figure}[h]
	\centering
	\begin{circuitikz}
		\node[connector] (con1) at (0,0)
		{\connectord{SMA\female}{SMA\female}{stříbrný}};
		\draw (con1.west) to[short, -o] +(-1,0);
		\draw (con1.east) to[short, -o] +(1,0);
	\end{circuitikz}
	\caption{Schéma zapojení 2. měření.}
	\label{fig:exp2}
\end{figure}

\begin{figure}[htp]
	\centering
	\input{plots/data01-s11}
	\input{plots/data01-s21}
	\caption{Naměřené hodnoty parametrů $\sparam_{11}$ a~$\sparam_{21}$
		zlaté spojky SMA\female.}
	\label{fig:01-sparam}
\end{figure}

\begin{figure}[htp]
	\centering
	\input{plots/data02-s11}
	\input{plots/data02-s21}
	\caption{Naměřené hodnoty parametrů $\sparam_{11}$ a~$\sparam_{21}$
		stříbrného spojky SMA\female.}
	\label{fig:02-sparam}
\end{figure}

\clearpage
\subsection{Měření 3: kaskáda spojek SMA}
Měříme zapojení
(SMAf/SMAf)$_\text{Z}$---(SMAm/SMAm)$_\text{Z}$---(SMAf/SMAf)$_\text{S}$,
kde m značí samčí konektor.
Sché\-ma zapojení je na obr.~\ref{fig:exp3}.
Úkolem je dopočítat matice prostřední spojky (SMAm/SMAm)$_\text{Z}$,
matice pro zbylé spojky známe z~předchozích měření.
Hodnoty $\sparam_{11}$ a~$\sparam_{21}$ pro naměřenou trojici spojek vidíme
na obr.~\ref{fig:03-sparam} a hodnoty dopočtené pomocí rovnic
\eqref{eq:tmatrix-elema} až \eqref{eq:s21} jsou vyneseny do grafu na
obr.~\ref{fig:03-result-sparam}. Z~dopočítaných hodnot vychází, že spojka by
měla při frekvencích $>\SI{1}{\giga\hertz}$ zesilovat signál. Jelikož se jedná
pouze o~velmi nízké hodnoty, vysvětlením může být šum nebo nedokonalé dotažení
spojů, na které je vysokofrekvenční měření citlivé.

\begin{figure}[h]
	\centering
	\begin{circuitikz}
		\node[connector] (con1) at (-4,0)
		{\connectord{SMA\female}{SMA\female}{zlatý}};
		\node[connector] (con2) at (0,0)
		{\connectord{SMA\male}{SMA\male}{zlatý}};
		\node[connector] (con3) at (4,0)
		{\connectord{SMA\female}{SMA\female}{stříbrný}};
		\draw (con1.west) to[short, -o] +(-1,0);
		\draw (con3.east) to[short, -o] +(1,0);
	\end{circuitikz}
	\caption{Schéma zapojení 3. měření.}
	\label{fig:exp3}
\end{figure}

\subsection{Měření 4: dvojice redukcí \connector{SMA\female}{BNC}}
Máme zapojení (SMAf/BNCm)---(BNCf/SMAf),
jeho schéma je vidět na obr.~\ref{fig:exp4}.
Jelikož nedokážeme přímo měřit ani jednu z~těchto redukcí samostatně
a chceme znát matici jedné z~nich,
redukce budeme považovat za identické a výslednou matici
jedné redukce určíme jako odmocninu z~naměřené matice.
Naměřené hodnoty parametrů $\sparam_{11}$ a~$\sparam_{21}$ pro celé zapojení
jsou vyneseny do grafů na obr.~\ref{fig:04-sparam} a~vypočítané hodnoty jsou na
obr.~\ref{fig:04-result-sparam}. V~porovnání s~SMA spojkami dochází u~těchto
redukcí k~většímu odrazu, a to až o~$\SI{2}{dB}$ při $\SI{5}{\giga\hertz}$.
Pokud si jako kritérium zvolíme $|\sparam_{11}| < \SI{-20}{dB}$,
je spojka vhodná pro použití do $\SI{2}{\giga\hertz}$.

\begin{figure}[h]
	\centering
	\begin{circuitikz}
		\node[connector] (con1) at (-2,0)
		{\connector{SMA\female}{BNC\male}};
		\node[connector] (con2) at (2,0)
		{\connector{BNC\female}{SMA\female}};
		\draw (con1.west) to[short, -o] +(-1,0);
		\draw (con2.east) to[short, -o] +(1,0);
	\end{circuitikz}
	\caption{Schéma zapojení 4. měření.}
	\label{fig:exp4}
\end{figure}

\subsection{Měření 5: spojka BNC{\female} -- BNC{\female}}
Ze zapojení (SMAf/BNCm)---(BNCf/BNCf)---(BNCm/SMAf),
jehož schéma je vidět na obr.~\ref{fig:exp5},
chceme nalézt parametry spojky (BNCf/BNCf).
Výpočet lze provézt díky znalosti výsledků z~mě\-ře\-ní č.~4.
Naměřené hodnoty parametrů $\sparam_{11}$ a~$\sparam_{21}$
jsou na obr.~\ref{fig:05-sparam}
a~vypočítané hodnoty parametrů $\sparam_{11}$ a~$\sparam_{21}$
pro spojku BNC\female jsou na obr.~\ref{fig:05-result-sparam}.
Opět vidíme, že s kritériem $|\sparam_{11}| < \SI{-20}{dB}$
není spojka pro frekvence nad $\SI{2}{\giga\hertz}$ vhodná.

\begin{figure}[h]
	\centering
	\begin{circuitikz}
		\node[connector] (con1) at (-4,0)
		{\connector{SMA\female}{BNC\male}};
		\node[connector] (con2) at (0,0)
		{\connector{BNC\female}{BNC\female}};
		\node[connector] (con3) at (4,0)
		{\connector{BNC\male}{SMA\female}};
		\draw (con1.west) to[short, -o] +(-1,0);
		\draw (con3.east) to[short, -o] +(1,0);
	\end{circuitikz}
	\caption{Schéma zapojení 5. měření.}
	\label{fig:exp5}
\end{figure}

\begin{figure}[p]
	\centering
	\input{plots/data03-s11}
	\input{plots/data03-s21}
	\caption{Naměřené hodnoty parametrů $\sparam_{11}$ a~$\sparam_{21}$
		kaskády spojek SMA.}
	\label{fig:03-sparam}
\end{figure}

\begin{figure}[p]
	\centering
	\input{plots/results03-s11}
	\input{plots/results03-s21}
	\caption{Spočtené hodnoty parametrů $\sparam_{11}$ a~$\sparam_{21}$
		spojky SMA\male.}
	\label{fig:03-result-sparam}
\end{figure}

\begin{figure}[p]
	\centering
	\input{plots/data04-s11}
	\input{plots/data04-s21}
	\caption{Naměřené hodnoty parametrů $\sparam_{11}$ a~$\sparam_{21}$
		dvojice redukcí \connector{SMA\female}{BNC}.}
	\label{fig:04-sparam}
\end{figure}

\begin{figure}[p]
	\centering
	\input{plots/results04-s11}
	\input{plots/results04-s21}
	\caption{Odhadnuté hodnoty parametrů $\sparam_{11}$ a~$\sparam_{21}$
		samotné redukce \connector{SMA\female}{BNC}.}
	\label{fig:04-result-sparam}
\end{figure}

\begin{figure}[p]
	\centering
	\input{plots/data05-s11}
	\input{plots/data05-s21}
	\caption{Naměřené hodnoty parametrů $\sparam_{11}$ a~$\sparam_{21}$
		sestavy redukcí \connector{SMA\female}{BNC} a~spojky BNC\female.}
	\label{fig:05-sparam}
\end{figure}

\begin{figure}[p]
	\centering
	\input{plots/results05-s11}
	\input{plots/results05-s21}
	\caption{Spočtené hodnoty parametrů $\sparam_{11}$ a~$\sparam_{21}$
		spojky BNC\female.}
	\label{fig:05-result-sparam}
\end{figure}

\clearpage
\subsection{Měření 6: rezonance na T kusu BNC}
\newcommand\freelen{d}
Zapojení (SMAf/BNCm)---(BNCf/BNCm/BNCf)---(BNCm/SMAf),
přičemž (BNCf/BNCm/BNCf) je T kus.
Prostřední port členu T je zde jako otevřený obvod, viz obr.~\ref{fig:exp6}.
Na otevřeném obvodu dochází k~odrazu signálu a destruktivní interferenci, což
je vidět na naměřených hodnotách parametrů $\sparam_{11}$ a~$\sparam_{21}$ na
obr.~\ref{fig:06-sparam}.
Délka prostředního portu $d$ je s~frekvencí $f$,
při níž dochází k~destruktivní interferenci, spojena vztahem
\begin{equation}
	d = \frac{(2k+1)}{4f} \frac{1}{\sqrt{\varepsilon_0 \varepsilon_r \mu_0}}
	\label{eq:resonance-length}
\end{equation}
kde k~= 0,1,2,... je řád maxima, $\varepsilon_0$ je permitivita vakua,
$\varepsilon_r = \num{2.25}$ je v~tomto případě relativní permitivita polyethylenu
a $\mu_0$ je permeabilita vakua.

V~obecném případě, kdy jsou známa více než dvě interferenční maxima,
lze pro stanovení neznámé délky $\freelen$ přepsat
vztah \eqref{eq:resonance-length} do tvaru:
\begin{align}
	\label{eq:resonance-length-freq}
	f &= \frac{1}{\freelen} \frac{2k+1}{4}
		\frac{1}{\sqrt{\permitfree\permitrel\permeabfree}}
		= ax + b &
	a &= 1/\freelen \\
	& &
	x &= \frac{2k+1}{4}
		\frac{1}{\sqrt{\permitfree\permitrel\permeabfree}}
\end{align}
aproximovat parametry $a$ a $b$ metodou nejmenších čtverců
a~spočítat délku $\freelen = 1/a$.

Jak je patrno z~obrázku~\ref{fig:06-sparam}, v~měřené oblasti se vyskytla
pouze dvě maxima a výše uvedený postup se proto redukoval na triviální
spojení dvou bodů přímkou.
Destruktivní interference pro $\wavelen/4$ nastala při frekvenci
\SI{3.33}{\giga\hertz} s~útlumem \SI{42}{\decibel}.
Pro $3\wavelen/4$ byl útlum \SI{36.4}{\decibel} při frekvenci
\SI{8.47}{\giga\hertz}.
Délka $d$ spočtená podle vztahu~\eqref{eq:resonance-length-freq} je
\SI{19.4}{\milli\metre}. Odhadovaná délka naměřená posuvným měřítkem je
$\approx\SI{23}{\milli\metre}$.
% TODO: Calculate uncertainty

\begin{figure}[hb]
	\centering
	\begin{circuitikz}[scale=0.9, every node/.style={scale=0.9}]
		\node[connector, minimum height=1.5cm] (con1) at (-4,0)
		{\connector{SMA\female}{BNC\male}};
		\draw (-2,-0.75) -- (2,-0.75) -- (2,0.75) -- (0.75,0.75) -- (0.75,2)
		-- (-0.75,2) -- (-0.75,0.75) -- (-2,0.75) -- cycle;
		\node at (-1.2,0) {BNC\female};
		\node at (0,1.375) {BNC\male};
		\node at (1.2,0) {BNC\female};
		\node[connector, minimum height=1.5cm] (con3) at (4,0)
		{\connector{BNC\male}{SMA\female}};
		\draw (con1.west) to[short, -o] +(-1,0);
		\draw (con3.east) to[short, -o] +(1,0);
	\end{circuitikz}
	\caption{Schéma zapojení 6. měření.}
	\label{fig:exp6}
\end{figure}

\subsection{Měření 7: rezonance na T kusu s~volným vedením}
Měříme (SMAf/BNCm)---(BNCf/BNCm + kabel /BNCf)---(BNCm/SMAf).
Zapojení je stejné jako u~měření 6, tentokrát je však na prostřední port
T kusu připojen kabel neznámé délky, viz obr.~\ref{fig:exp7}. Naměřené hodnoty
$\sparam_{11}$ a~$\sparam_{21}$ jsou na obr.~\ref{fig:07-sparam}.

Postup je obdobný předchozímu případu. Pro výpočet délky $\freelen$
jsme zvolili minima signálu $\sparam_{21}$ příslušící frekvencím
pod \SI{3e8}{\hertz}, tedy patnáct hodnot.
U~vyšších frekvencí roste nejistota určení polohy minima, jelikož
se prodlužuje krok vzorkování, a proto nebyly uvažovány.

Délka kabelu spočtená výše uvedenou metodou činí
$\freelen = \SI{5.072(4)}{\metre}$.

\begin{figure}[hb]
	\centering
	\begin{circuitikz}[scale=0.9, every node/.style={scale=0.9}]
		\node[connector, minimum height=1.5cm] (con1) at (-4,0)
		{\connector{SMA\female}{BNC\male}};
		\draw (-2,-0.75) -- (2,-0.75) -- (2,0.75) -- (0.75,0.75) -- (0.75,2)
		-- (-0.75,2) -- (-0.75,0.75) -- (-2,0.75) -- cycle;
		\node at (-1.2,0) {BNC\female};
		\node at (0,1.375) {BNC\male};
		\node at (1.2,0) {BNC\female};
		\node[connector, minimum height=1.5cm] (con3) at (4,0)
		{\connector{BNC\male}{SMA\female}};
		\draw (0,2) to[short, -o] (0,2.5);
		\draw (con1.west) to[short, -o] +(-1,0);
		\draw (con3.east) to[short, -o] +(1,0);
	\end{circuitikz}
	\caption{Schéma zapojení 7. měření.}
	\label{fig:exp7}
\end{figure}

\begin{figure}[p]
	\centering
	\input{plots/data06-s11}
	\input{plots/data06-s21}
	\caption{Naměřené hodnoty parametrů $\sparam_{11}$ a~$\sparam_{21}$
		sestavy redukcí \connector{SMA\female}{BNC} a~T kusu BNC.}
	\label{fig:06-sparam}
\end{figure}
\clearpage

\begin{figure}[p]
	\centering
	\input{plots/data07-s11}
	\input{plots/data07-s21}
	\caption{Naměřené hodnoty parametrů $\sparam_{11}$ a~$\sparam_{21}$
		sestavy redukcí \connector{SMA\female}{BNC}
		a~T kusu BNC s~volným vedením.}
	\label{fig:07-sparam}
\end{figure}

\clearpage
\subsection{Měření 8: rezonance na T kusu s~vedením zakončeným terminátorem}
Zapojení (SMAf/BNCm)---(BNCf/BNCm + kabel + terminátor /BNCf)---(BNCm/SMAf)
vychází z~měření 7, ke kabelu je navíc připojen terminátor.

Jelikož impedance terminátoru je shodná s~impedancí vedení,
z~pohledu vstupujícího signálu se jedná o~dvě stejně velké impedance
zapojené vedle sebe.
Dochází tedy k~dělení výkonu rovným dílem mezi obě větve.
Očekáváme proto, že výkon procházejícího signálu bude polovinou výkonu
vstupujícího, což odpovídá poklesu o~zhruba \SI{3.01}{\decibel}.
Jak je patrno z~parametru $\sparam_{21}$ na obr.~\ref{fig:08-sparam},
tento předpoklad je splněn.

\begin{figure}[h]
	\centering
	\begin{circuitikz}
		\node[connector, minimum height=1.5cm] (con1) at (-4,0)
		{\connector{SMA\female}{BNC\male}};
		\draw (-2,-0.75) -- (2,-0.75) -- (2,0.75) -- (0.75,0.75) -- (0.75,2)
		-- (-0.75,2) -- (-0.75,0.75) -- (-2,0.75) -- cycle;
		\node at (-1.2,0) {BNC\female};
		\node at (0,1.375) {BNC\male};
		\node at (1.2,0) {BNC\female};
		\node[connector, minimum height=1.5cm] (con3) at (4,0)
		{\connector{BNC\male}{SMA\female}};
		\node[genericshape, rotate=90] (term) at (0,3.5) {};
		\node at (0.6,3.5) {$\impedance_0$};
		\draw (0,2) -- (term.west);
		\draw (con1.west) to[short, -o] +(-1,0);
		\draw (con3.east) to[short, -o] +(1,0);
	\end{circuitikz}
	\caption{Schéma zapojení 8. měření.}
	\label{fig:exp8}
\end{figure}

\subsection{Měření 9: banánkový spoj}
Zapojení (SMAf/BNCm)---(BNCf/banánky m)---(banánky f/BNCf)---(BNCm/SMAf),
přičemž ba\-nán\-ky jsou koaxiální vedení zapojené způsobem jádro--jádro a
stínění--stínění, viz schéma na obr.~\ref{fig:exp9}. Z~obr.~\ref{fig:09-sparam}
můžeme říct, že pro kritérium $|\sparam_{11}| < \SI{-20}{dB}$ je tento typ spoje použitelný do $\SI{50}{\mega\hertz}$. Nad
$\SI{1}{\giga\hertz}$ je oproti předchozím měřením vidět, že
banánky nejsou vůbec vhodné pro vysoké frekvence.

\begin{figure}[h]
	\centering
	\begin{circuitikz}
		\node[connector] (con1) at (-5,0)
		{\connector{SMA\female}{BNC\male}};
		\node[connector, minimum width=1.4cm] (con2) at (-2.3,0)
		{BNC\female};
		\node[connector, minimum width=1.4cm] (con3) at (2.3,0)
		{BNC\female};
		\coordinate[yshift=2mm] (n1) at (con2.east) {};
		\coordinate[yshift=0-2mm] (n2) at (con2.east) {};
		\coordinate[yshift=2mm] (n3) at (con3.west) {};
		\coordinate[yshift=0-2mm] (n4) at (con3.west) {};
		\draw (n1) to[short, -o] +(1.4,0);
		\draw (n2) to[short, -*] +(1.4,0);
		\draw (n3) to[short, -o] +(-1.4,0);
		\draw (n4) to[short, -*] +(-1.4,0);
		\node at (0,0.6) {banánky};
		\node[connector] (con4) at (5,0)
		{\connector{BNC\male}{SMA\female}};
		\draw (con1.west) to[short, -o] +(-1,0);
		\draw (con4.east) to[short, -o] +(1,0);
	\end{circuitikz}
	\caption{Schéma zapojení 9. měření.}
	\label{fig:exp9}
\end{figure}

\subsection{Měření 10: banánkový spoj s~překřížením}
Vycházíme ze zapojení jako u~měření 9 (SMAf/BNCm)---(BNCf/banánky m)---(banánky
f/BNCf)---(BNCm/SMAf), přičemž nyní přehozením banánků máme zapojení
koaxiálního vedení do kříže, tedy jádro--stínění, viz obr.~\ref{fig:exp10}.

Očekáváme horší výsledky než v~měření 9. Na obr.~\ref{fig:10-sparam} pozorujeme nevyhovující hodnoty parametrů $\sparam_{11}$ a~$\sparam_{21}$ na celém spektru
frekvencí.

\begin{figure}[h]
	\centering
	\begin{circuitikz}
		\node[connector] (con1) at (-5,0)
		{\connector{SMA\female}{BNC\male}};
		\node[connector, minimum width=1.4cm] (con2) at (-2.3,0)
		{BNC\female};
		\node[connector, minimum width=1.4cm] (con3) at (2.3,0)
		{BNC\female};
		\coordinate[yshift=2mm] (n1) at (con2.east) {};
		\coordinate[yshift=0-2mm] (n2) at (con2.east) {};
		\coordinate[yshift=2mm] (n3) at (con3.west) {};
		\coordinate[yshift=0-2mm] (n4) at (con3.west) {};
		\draw (n1) to[short, -o] +(1.4,0);
		\draw (n2) to[short, -*] +(1.4,0);
		\draw (n3) to[short, -*] +(-1.4,0);
		\draw (n4) to[short, -o] +(-1.4,0);
		\node at (0,0.6) {banánky};
		\node[connector] (con4) at (5,0)
		{\connector{BNC\male}{SMA\female}};
		\draw (con1.west) to[short, -o] +(-1,0);
		\draw (con4.east) to[short, -o] +(1,0);
	\end{circuitikz}
	\caption{Schéma zapojení 10. měření.}
	\label{fig:exp10}
\end{figure}

\begin{figure}[p]
	\centering
	\input{plots/data08-s11}
	\input{plots/data08-s21}
	\caption{Naměřené hodnoty parametrů $\sparam_{11}$ a~$\sparam_{21}$
		sestavy redukcí \connector{SMA\female}{BNC}
		a~T kusu BNC s~ vedením a~terminátorem.}
	\label{fig:08-sparam}
\end{figure}

\begin{figure}[p]
	\centering
	\input{plots/data09-s11}
	\input{plots/data09-s21}
	\caption{Naměřené hodnoty parametrů $\sparam_{11}$ a~$\sparam_{21}$
		sestavy redukcí \connector{SMA\female}{BNC}
		a~\connector{BNC}{banánky}.}
	\label{fig:09-sparam}
\end{figure}

\begin{figure}[p]
	\centering
	\input{plots/data10-s11}
	\input{plots/data10-s21}
	\caption{Naměřené hodnoty parametrů $\sparam_{11}$ a~$\sparam_{21}$
		sestavy redukcí \connector{SMA\female}{BNC}
		a~\connector{BNC}{banánky} zapojených křížem.}
	\label{fig:10-sparam}
\end{figure}

\clearpage
\subsection{Měření 11: krokodýlky}
Zapojení (SMAf/BNCm)---(BNCf/banánky m)---(krokodýlky na prázdno) jako
jednobran bez při\-pojeného přijímače, viz schéma na obr.~\ref{fig:exp11}.
Toto měření předchází měřením č. 12 a 13, abychom zjistili hodnoty
$\sparam_{11}$ tohoto vedení, jehož naměřené hodnoty viz
na~obr.~\ref{fig:11-sparam}.
K~odrazu dochází při frekvencích $\SI{0.1}{\giga\hertz}$ a výše.

\begin{figure}[h]
	\centering
	\begin{circuitikz}
		\node[connector] (con1) at (-5,0)
		{\connector{SMA\female}{BNC\male}};
		\node[connector, minimum width=1.4cm] (con2) at (-2.3,0)
		{BNC\female};
		\coordinate[yshift=0-2mm] (n2) at (con2.east) {};
		\draw (con2.east)++(0,0.2) to[short, -o] ++(1.4,0) -- ++(0,0.5)
		-- node[at end] (clip1) {} ++(2,0);
		\draw (con2.east)++(0,-0.2) to[short, -*] ++(1.4,0) -- ++(0,-0.5)
		-- node[at end] (clip2) {} ++(2,0);
		\draw[very thick] (clip1.center) -- +(150:1);
		\draw (clip1.center) +(-0.7,0) arc (180:150:0.7);
		\draw[very thick] (clip2.center) -- +(150:1);
		\draw (clip2.center) +(-0.7,0) arc (180:150:0.7);
		\node[yshift=1cm] at (clip2) {krokodýlky};
		\draw (con1.west) to[short, -o] +(-1,0);
	\end{circuitikz}
	\caption{Schéma zapojení 11. měření.}
	\label{fig:exp11}
\end{figure}

\begin{figure}[htp]
	\centering
	\input{plots/data11-s11}
	\caption{Naměřené hodnoty parametru $\sparam_{11}$
		sestavy redukcí a~volných krokodýlků.}
	\label{fig:11-sparam}
\end{figure}

\clearpage
\subsection{Měření 12: krokodýlky s~cívkou}
Zapojení (SMAf/BNCm)---(BNCf/banánky m)---(krokodýlky + cívka) jako jednobran,
viz schéma na obr.~\ref{fig:exp12}. Úkolem je spočítat z~naměřených hodnot
parametru $\sparam_{11}$, které jsou vidět na obr.~\ref{fig:12-sparam},
indukčnost cívky $\inductance$. Ta lze spočítat ze vztahu
\begin{equation}
	\inductance = \frac{\impedance}{i2\pi\freq}
	\label{eq:inductance}
\end{equation}

Impedanci $\impedance$ spočítáme pomocí rovnice \eqref{eq:Z}. Závislost
impedance $\impedance$ cívky na frekvenci je na obr.~\ref{fig:12-result-z} a
závislost indukčnosti $\inductance$ cívky na frekvenci je na
obr.~\ref{fig:12-result-l}. Pro nízké frekvence se cívka chová běžně, naměřená
hodnota $\inductance = \SI{1.63e-4}{\henry}$, ale s~rostoucí
frekvencí klesá zdánlivá indukčnost.

\begin{figure}[h]
	\centering
	\begin{circuitikz}
		\node[connector] (con1) at (-5,0)
		{\connector{SMA\female}{BNC\male}};
		\node[connector, minimum width=1.4cm] (con2) at (-2.3,0)
		{BNC\female};
		\coordinate[yshift=0-2mm] (n2) at (con2.east) {};
		\draw (con2.east)++(0,0.2) to[short, -o] ++(1.4,0) -- ++(0,0.5)
		-- node[midway] (clip1) {} ++(4,0) to[inductor]
		++(0,-1.4) -- node[midway] (clip2) {} ++(-4,0) -- ++(0,0.5);
		\draw (con2.east)++(0,-0.2) to[short, -*] +(1.4,0);
		\draw[very thick] (clip1.center) -- +(150:1);
		\draw (clip1.center) +(-0.7,0) arc (180:150:0.7);
		\draw[very thick] (clip2.center) -- +(150:1);
		\draw (clip2.center) +(-0.7,0) arc (180:150:0.7);
		\node[yshift=1cm] at (clip2) {krokodýlky};
		\draw (con1.west) to[short, -o] +(-1,0);
	\end{circuitikz}
	\caption{Schéma zapojení 12. měření.}
	\label{fig:exp12}
\end{figure}

\begin{figure}[hb]
	\centering
	\input{plots/data12-s11}
	\caption{Naměřené hodnoty parametru $\sparam_{11}$
		sestavy redukcí a~krokodýlků s~cívkou.}
	\label{fig:12-sparam}
\end{figure}

\begin{figure}[p]
	\centering
	\input{plots/results12-z}
	\caption{Spočtené hodnoty impedance $\impedance$ cívky.}
	\label{fig:12-result-z}
\end{figure}

\begin{figure}[p]
	\centering
	\input{plots/results12-l}
	\caption{Spočtené hodnoty indukčnosti $\inductance$ cívky.}
	\label{fig:12-result-l}
\end{figure}

\clearpage
\subsection{Měření 13: krokodýlky s~kondenzátorem}
Zapojení (SMAf/BNCm)---(BNCf/banánky m)---(krokodýlky s~kondenzátorem) jako
jednobran, viz schéma na
obr.~\ref{fig:exp13}. Úkolem je spočítat z~naměřených hodnot parametru
$\sparam_{11}$, které jsou vidět na obr.~\ref{fig:13-sparam}, kapacitu
kondenzátoru $\capacitance$. Ta lze spočítat ze vztahu

\begin{equation}
	\capacitance = \frac{1}{i2\pi\freq\impedance}
	\label{eq:capacitance}
\end{equation}

Impedanci $\impedance$ spočítáme opět pomocí rovnice \eqref{eq:Z}. Závislost
impedance $\impedance$ kondenzátoru na frekvenci je na
obr.~\ref{fig:13-result-z}. Kapacita $\capacitance$ klesá s~rostoucí frekvencí,
tuto závislost vidíme na obr.~\ref{fig:13-result-c}. Pro frekvence $\freq =
10^4$~Hz je vypočítaná hodnota kapacity $\capacitance =
10^{-8}$~F.

\begin{figure}[h]
	\centering
	\begin{circuitikz}
		\node[connector] (con1) at (-5,0)
		{\connector{SMA\female}{BNC\male}};
		\node[connector, minimum width=1.4cm] (con2) at (-2.3,0)
		{BNC\female};
		\coordinate[yshift=0-2mm] (n2) at (con2.east) {};
		\draw (con2.east)++(0,0.2) to[short, -o] ++(1.4,0) -- ++(0,0.5)
		-- node[midway] (clip1) {} ++(4,0) to[capacitor]
		++(0,-1.4) -- node[midway] (clip2) {} ++(-4,0) -- ++(0,0.5);
		\draw (con2.east)++(0,-0.2) to[short, -*] +(1.4,0);
		\draw[very thick] (clip1.center) -- +(150:1);
		\draw (clip1.center) +(-0.7,0) arc (180:150:0.7);
		\draw[very thick] (clip2.center) -- +(150:1);
		\draw (clip2.center) +(-0.7,0) arc (180:150:0.7);
		\node[yshift=1cm] at (clip2) {krokodýlky};
		\draw (con1.west) to[short, -o] +(-1,0);
	\end{circuitikz}
	\caption{Schéma zapojení 13. měření.}
	\label{fig:exp13}
\end{figure}

\begin{figure}[hb]
	\centering
	\input{plots/data13-s11}
	\caption{Naměřené hodnoty parametru $\sparam_{11}$
		sestavy redukcí a~krokodýlků s~kondenzátorem.}
	\label{fig:13-sparam}
\end{figure}

\begin{figure}[p]
	\centering
	\input{plots/results13-z}
	\caption{Spočtené hodnoty impedance $\impedance$ kondenzátoru.}
	\label{fig:13-result-z}
\end{figure}

\begin{figure}[p]
	\centering
	\input{plots/results13-c}
	\caption{Spočtené hodnoty kapacity $\capacitance$ kondenzátoru.}
	\label{fig:13-result-c}
\end{figure}

\clearpage
\subsection{Měření 14: banánky s~vodičem}
Zapojení (SMAf/BNCf)---(BNCm/kabel/banánky) jako jednobran, viz schéma na
obr.~\ref{fig:exp14}. Naměřené hodnoty $\sparam_{11}$ parametru jsou na
obr.~\ref{fig:14-sparam}. Toto měření předchází následujícímu měření, abychom
znali vlastnosti tohoto vedení.

\begin{figure}[h]
	\centering
	\begin{circuitikz}
		\node[connector] (con1) at (-5,0)
		{\connector{SMA\female}{BNC\male}};
		\node[connector, minimum width=1.4cm] (con2) at (-2.3,0)
		{BNC\female};
		\coordinate[yshift=0-2mm] (n2) at (con2.east) {};
		\draw (con2.east)++(0,0.2) -- ++(1.4,0) -- ++(0,0.5)
		to[short, -o] node[at end] (plug1) {} ++(2,0);
		\draw (con2.east)++(0,-0.2) -- ++(1.4,0) -- ++(0,-0.5)
		to[short, -*] node[at end] (plug2) {} ++(2,0);
		\draw (con1.west) to[short, -o] +(-1,0);
	\end{circuitikz}
	\caption{Schéma zapojení 14. měření.}
	\label{fig:exp14}
\end{figure}

\subsection{Měření 15: banánky s~vodičem a~rezistorem}
Zapojení je stejné jako v~předchozím měření s~rezistorem připojeným na banánky
(SMAf/BNCf)---(BNCm/kabel/banánky)---(rezistor), viz schéma
na obr.~\ref{fig:exp15}. Úkolem je spočítat odpor rezistoru.
V~oblasti, kde je imaginární složka impedance zanedbatelná,
můžeme odpor položit roven reálné složce impedance:
\begin{equation}
	\resistance = \real(\impedance)
\end{equation}

Naměřené hodnoty parametru $\sparam_{11}$ jsou na obr.~\ref{fig:15-sparam}.
Spočtené hodnoty odporu $\impedance$ jsou na obr.~\ref{fig:15-result-z}.
Výsledná hodnota $\SI{122}{\ohm}$ je konstantní až do frekvencí
$\SI{e7}{\hertz}$, kde začnou mít vliv rozměry rezistoru i~přívody a vzniká parazitní
kapacita a indukčnost. Stejnosměrným multimetrem jsme naměřili hodnotu
$\SI{125}{\ohm}$.

\begin{figure}[h]
	\centering
	\begin{circuitikz}
		\node[connector] (con1) at (-5,0)
		{\connector{SMA\female}{BNC\male}};
		\node[connector, minimum width=1.4cm] (con2) at (-2.3,0)
		{BNC\female};
		\coordinate[yshift=0-2mm] (n2) at (con2.east) {};
		\draw (con2.east)++(0,0.2) -- ++(1.4,0) -- ++(0,1)
		to[short, -o] ++(2,0)
		to[R, l=$\resistance$] ++(0,-2.4)
		to[short, *-] ++(-2,0) -- ++(0,1) -- ++(-1.4,0);
		\draw (con1.west) to[short, -o] +(-1,0);
	\end{circuitikz}
	\caption{Schéma zapojení 15. měření.}
	\label{fig:exp15}
\end{figure}

\subsection{Měření 16: BNC redukce zakončená terminátorem}
Zapojení (SMAf/BNCf)---(BNCm/terminátor) jako jednobran, viz schéma na
obr.~\ref{fig:exp16}. Úkolem je změřit impedanci $\impedance_{T}$ terminátoru.
Naměřené hodnoty parametru $\sparam_{11}$ jsou na obr.~\ref{fig:16-sparam11}.
Vypočítané hodnoty impedance $\impedance_{T}$ jsou na
obr.~\ref{fig:16-result-z}. Výsledná hodnota
$\SI{51}{\ohm}$ je konstantní až do frekvencí $10^9$~Hz a odpovídá impedanci terminátoru naměřené multimetrem.

\begin{figure}[h]
	\centering
	\begin{circuitikz}
		\node[connector] (con1) at (-5,0)
		{\connector{SMA\female}{BNC\male}};
		\node[connector, minimum width=1.4cm] (con2) at (-2.3,0)
		{BNC\female};
		\node[genericshape,label=below:$\impedance_0$] (term) {};
		\draw (con2.east) -- (term.west);
		\draw (con1.west) to[short, -o] +(-1,0);
	\end{circuitikz}
	\caption{Schéma zapojení 16. měření.}
	\label{fig:exp16}
\end{figure}

\begin{figure}[p]
	\centering
	\input{plots/data14-s11}
	\caption{Naměřené hodnoty parametru $\sparam_{11}$
		sestavy redukcí a~vodiče s~banánky.}
	\label{fig:14-sparam}
\end{figure}

\begin{figure}[p]
	\centering
	\input{plots/data15-s11}
	\caption{Naměřené hodnoty parametru $\sparam_{11}$
		sestavy redukcí a~vodiče s~banánky a~rezistorem.}
	\label{fig:15-sparam}
\end{figure}

\begin{figure}[p]
	\centering
	\input{plots/results15-z}
	\caption{Spočtené hodnoty impedance $\impedance$ banánků s~rezistorem.}
	\label{fig:15-result-z}
\end{figure}

\begin{figure}[p]
	\centering
	\input{plots/data16-s11}
	\caption{Naměřené hodnoty parametru $\sparam_{11}$ BNC redukce zakončené
	terminátorem.}
	\label{fig:16-sparam11}
\end{figure}

\begin{figure}[tp]
	\centering
	\input{plots/results16-z}
	\caption{Spočtené hodnoty impedance $\impedance$
		konektoru BNC s~terminátorem.}
	\label{fig:16-result-z}
\end{figure}

\clearpage
\subsection{Měření 17: obvod s~vlnovodem}
\newcommand\wavelencrit{\wavelen_\mathrm{krit}}
Zapojení (SMAf/SMAf)---(vlnovod)---(SMAm/SMAm), které je vidět na
obr.~\ref{fig:exp17}. Pro vlnovod existuje tzv. kritická vlnová délka
$\wavelencrit$. Do vlnovodu může vstoupit pouze vlna kratší vlnové délky než
$\wavelencrit$, která je dána delší stranou vlnovodu $a$:
\begin{equation}
	\wavelencrit = 2a
	\label{eq:vlnovod}
\end{equation}

Dle naměřených hodnot $\sparam_{21}$, které vidíme na
obr.~\ref{fig:17-sparam21}, pozorujeme na frekvenci $\SI{1.75}{\giga\hertz}$
nárůst signálu. Tato frekvence odpovídá vlnové délce $\wavelencrit = c/f =
\SI{171}{\milli\metre}$. Delší stranu $a = \SI{85.5}{\milli\metre}$,
vypočtenou dle rovnice \eqref{eq:vlnovod}, lze porovnat s~ručně naměřenou
vnitřní šířkou vlnovodu -- $\SI{85}{\milli\metre}$.

\begin{figure}[h]
	\centering
	\begin{circuitikz}
		\node[connector] (con1) at (-5,0)
		{\connector{SMA\female}{SMA\female}};
		\node[draw, align=center,minimum width=6cm, minimum height=2cm]
		(con2) at (0,0)
		{vlnovod};
		\node[connector] (con3) at (5,0)
		{\connector{SMA\male}{SMA\male}};
		\draw (con1.west) to[short, -o] +(-1,0);
		\draw (con3.east) to[short, -o] +(1,0);
	\end{circuitikz}
	\caption{Schéma zapojení 17. měření.}
	\label{fig:exp17}
\end{figure}

\begin{figure}[htbp]
	\centering
	\input{plots/data17-s21}
	\caption{Naměřené hodnoty parametru $\sparam_{21}$ sestavy s~vlnovodem.}
	\label{fig:17-sparam21}
\end{figure}

\newpage
\section{Závěr}
V~této úloze jsme se seznámili s~vektorovým síťovým analyzátorem, pomocí nejž
jsme zjišťovali vhodnost různých typů spojek a redukcí pro vysokofrekvenční
využití. Měřili jsme prvky rozptylové matice různých obvodů, z~nichž jsme
v~některých případech určovali rozptylové matice některého z~neznámých prvků
v~obvodu. Dále jsme dopočítávali fyzikální veličiny jako impedanci, indukčnost
cívky, kapacitu kondenzátoru, odpor rezistoru nebo kritickou vlnovou délku
vlnovodu.
Zjistili jsme, že SMA lze použít pro vyšší frekvence než BNC. Dále se
u~vysokých frekvencí některé prvky nechovají standardně. Rezistor již při
frekvenci $10^7$~Hz obsahuje parazitní kapacitu a indukčnost a pro
vyšší frekvence je potřeba jeho miniaturizace. Podobně také cívka ztrácí své vlastnosti s~rostoucí
frekvencí, projevuje se u ní mezizávitová kapacita. V měření 13 došlo k chybě v podobě záměny prvků a ve skutečnosti není měřený prvek kondenzátor, ale termistor.

\newpage
\printbibliography

\end{document}
